\section{Putting It All Together}

Now let's take these two previous ideas and put them together. How could we draw an exact sample from the standard normal distribution? At this point it's simple. First, use exact bisection sampling and perform a random walk on the integers, starting at 0. As soon as the sample has been narrowed down to an interval of unit length, the integer part of the sample has been determined. Then use exact rejection sampling to determine the fractional part. That's all there is to it. We can now easily produce the digits of an exactly standard normally distributed sample to our hearts content.\\

At this point it seems worth reflecting a little more generally. First, because we're eventually just passing through random bits, this algorithm can easily be seen to be asymptotically optimal. We'll examine the exact meaning of "eventually" in section 7. For now, let's direct our attention to the bits required to sample $x$ during the rejection sampling stage. A few of these bits will be discarded while we iterate the rejection method. But the rest is then returned completely unmodified. A different way of looking at this is that we're taking a sample from the uniform distribution and sawing off a little piece at the front. The tail end is then precisely the fractional part of a random variable having practically any desired continuous distribution.

Furthermore, we could have used the exact same sequence to also draw an exact sample from some other unrelated distribution. These two samples would then be correlated in the curious way that, at some offset, eventually all of their digits coincide. Stated differently, if one looks at a sample from the uniform distribution, chances are that, \textit{within eyesight}, one can see the superimposed beginnings of samples from any continuous distribution one can care to mention.

Finally, consider the following. After having determined the leading digits of a sample from some distribution, we could write them down on a slip of paper and pass it to somebody else. We'll call such string of digits a \textit{prefix}. That person wouldn't even have to know which distribution the prefix was drawn from. They would still hold in their hands an \textit{exact} sample. If they're unsatisfied with the precision of the prefix we gave them, they can just add their own random digits! In very much the same way that the symbol $\pi$ summarizes an infinite sequence of digits, so do the digits on that slip of paper.

As an aside, it is also possible, in a sense, to reconstitute a prefix once many random digits have already been attached. Though this may easily be a different prefix and one does now need the density $f$ of the distribution from which the sample is drawn. The idea is to simply perform rejection sampling using $f$ as a proposal density and the uniform as the target distribution. As soon as a uniform has been successfully drawn, one can be confident that the remaining, unexamined digits are uniform, thus the digits consumed so far constitute a prefix. One caveat here is that one has to condition and rescale $f$ every time a rejection takes place, because, unlike with the uniform distribution, the trailing digits of $f$ are unlikely to again be distributed according to $f$. While this would be a daunting challenge to program, considering one requires arbitrary precision, there is nothing that prevents this in principle.

Now it is tempting to refer to the distribution of numbers that we could write on that slip of paper as a \textit{prefix distribution}. There's just one slight wrinkle that we have to iron out before we're fully justified to make that designation. The problem is that distinct prefixes may denote the same real number. Take the prefixes $0$ and $00$. Both of these designate distinct prefixes. One cannot replace one with the other without altering the underlying distribution that one is sampling. But if forced to map these to real numbers one would end up on the same point on the number line.

The way out here is to realize that we're free to add some random bits of our own before passing along the slip of paper. So all we do is sample bits until we hit a $1$. At that point our prefix designates a unique real number and we terminate. In effect we've just done a little part of the other person's job. Let's refer to these elongated prefixes as \textit{padded prefixes} as opposed to \textit{unpadded prefixes} for their unmodified counterparts. If it is clear from context we'll also often just use the term \textit{prefix} for either version.